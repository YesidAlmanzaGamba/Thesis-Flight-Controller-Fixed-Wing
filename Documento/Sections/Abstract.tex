  \section{Abstract}

This thesis addresses the critical need for efficient data management and operational control in fixed-wing UAVs, targeting the gaps in current commercial flight controllers. The developed flight controller resolves issues related to data storage, retrieval, and bidirectional communication for flight preprogramming both in-flight and on the ground. Key technical features include support for up to 16 actuators, an alarm buzzer, and real-time telemetry to a ground station. The controller operates in two modes: auto-stabilized and fly-by-wire, and monitors critical variables such as temperature, pressure, altitude, yaw, pitch, roll, and GPS coordinates in real-time. The controller was tested in a fixed-wing UAV. The tests demonstrated the system's reliability and effectiveness in maintaining stable flight conditions and accurate data logging. It is important to highlight that all these improvements were embedded into a single system, making it innovative. A noteworthy aspect is the shift from the traditional STM32 architecture to the ESP32-S3, enhancing the system's capabilities. This innovation significantly enhances UAV data management, operational efficiency, and safety, impacting UAV operators and researchers by providing a robust, reliable, and efficient flight control system.\\ \\ \\


\begin{center}
    \textbf{\LARGE{Resumen}} 
\end{center}

\vspace{10 px}
Esta tesis aborda la necesidad crítica de una gestión eficiente de datos y control operativo en UAVs de ala fija, enfocándose en las brechas de los controladores de vuelo comerciales actuales. El controlador de vuelo desarrollado resuelve problemas relacionados con el almacenamiento de datos, recuperación y comunicación bidireccional para la preprogramación de vuelos tanto en vuelo como en tierra. Las características técnicas clave incluyen soporte para hasta 16 actuadores, un buzzer de alarma y telemetría en tiempo real a una estación terrestre. El controlador opera en dos modos: autoestabilizado y fly-by-wire, y monitorea en tiempo real variables críticas como temperatura, presión, altitud, yaw, pitch, roll y coordenadas GPS. El controlador fue probado en un UAV de ala fija. Las pruebas demostraron la fiabilidad y efectividad del sistema para mantener condiciones de vuelo estables y un registro de datos preciso. Es importante destacar que todas estas mejoras fueron embebidas en un solo sistema, lo que lo hace novedoso. Un aspecto notable es el cambio de la arquitectura tradicional STM32 a la ESP32-S3, mejorando las capacidades del sistema. Esta innovación mejora significativamente la gestión de datos, la eficiencia operativa y la seguridad de los UAVs, impactando a operadores e investigadores de UAVs al proporcionar un sistema de control de vuelo robusto, fiable y eficiente. 