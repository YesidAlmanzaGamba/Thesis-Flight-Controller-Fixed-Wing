\section{Planteamiento del problema}

En la actualidad, los controladores de vuelo enfrentan desafíos significativos en la gestión y control de Vehículos Aéreos No Tripulados (UAVs) debido a la complejidad y las exigencias de las operaciones en tiempo real. La investigación reciente en controladores de vuelo ha avanzado considerablemente, destacándose la implementación de sistemas de control en tiempo real (RTOS) que optimizan la gestión de tareas críticas y no críticas, mejorando la robustez y fiabilidad del sistema. Asimismo, la integración de sensores redundantes y técnicas de amortiguación mecánica han mejorado la precisión de las lecturas y la estimación de la actitud del UAV, asegurando la fiabilidad del vuelo en condiciones adversas. El uso de microcontroladores avanzados como el ARM Cortex-M ha permitido una optimización significativa en términos de rendimiento y capacidades de procesamiento en tiempo real.\\


A pesar de los avances tecnológicos, actualmente los controladores de vuelo presentan algunas deficiencias. En primera instancia, tienen un número de actuadores limitado que pueden controlar. Además, presentan una falta de indicadores visuales y auditivos para verificar el correcto funcionamiento del dispositivo previamente al vuelo. De igual forma, estos dispositivos no tienen incorporado un sistema de telemetría, sino que este se adiciona como un modulo externo. Además, se hace necesario explorar microcontroladores que tengan mayor poder de procesamiento que los que se usan en la actualidad. \\


Por tanto, se quiere optimizar y mejorar el sistema aéreo no tripulado mediante el desarrollo de un controlador de vuelo que ofrezca mayor fiabilidad y efectividad para mantener condiciones de vuelo estables, así como un monitoreo constante de la aeronave. Esto mediante la implementación de un mecanismo de redundancia en el sistema de movimiento inercial, así cómo indicadores de funcionamiento durante el vuelo. Adicionalmente, se cuenta con un sistema de telemetría que le permite al controlador a su vez ser un estación en tierra. Esto incluye la implementación de sistemas de control avanzados, integración de sensores de alta precisión, y el uso de microcontroladores de última generación, asegurando así una operación óptima para los UAVs en diversos entornos y condiciones de vuelo.