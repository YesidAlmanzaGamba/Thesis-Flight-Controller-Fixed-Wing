\section{Objetivos}

\subsection{Objetivo General}


Desarrollar un controlador de vuelo para UAVs de ala fija que priorice las condiciones de almacenamiento y recuperación de datos de vuelo. El controlador deberá soportar hasta 16 actuadores, incorporar indicadores de funcionamiento, y permitir la comunicación bidireccional para la programación de vuelos desde el aire o la tierra. El controlador también debe tener la capacidad de ser una Ground Control Station (CGS) para obtener datos en tierra.


\subsection{Objetivos Específicos}

\begin{enumerate}


\item \textbf{Diseño del Controlador:}

Diseñar un controlador de vuelo que cumpla con los requisitos para ser aplicado en un UAV de ala fija. Este proceso incluirá la selección y disposición de componentes electrónicos en la PCB, seguido de una fase de implementación que implicará la verificación del diseño a través del uso del dispositivo en condiciones reales de operación. 
\\
\item \textbf{Diseño de Firmware e Interfaces de Vuelo:}
    Desarrollar el Firmware necesario para el controlador de vuelo diseñado, incluyendo algoritmos de control, sistemas de comunicación con los sensores y actuadores del UAV, así como el software de las interfaces de vuelo para el monitoreo del sistema.
\\
\item \textbf{Implementación del Controlador de Vuelo en un UAV  de Ala Fija:}
Instalar y operar el controlador de vuelo diseñado en un UAV de ala fija real. Se llevará a cabo la integración física del controlador en la aeronave, realizando la conexión con los sensores y actuadores específicos del UAV de ala fija, se realizará una prueba de vuelo para verificar el funcionamiento del controlador en condiciones reales de operación.

\end{enumerate}