
\section{Estado del Arte}

En los últimos años, la investigación en controladores de vuelo para UAVs (Vehículos Aéreos No Tripulados) ha avanzado considerablemente, centrando sus esfuerzos en el desarrollo de sistemas integrados y eficientes. Una de las áreas más destacadas ha sido la implementación de sistemas de control en tiempo real (RTOS, por sus siglas en inglés), que facilitan una gestión eficiente tanto de las tareas críticas como de las no críticas dentro del controlador de vuelo. Sistemas como el propuesto por Rico et al., hacen uso de arquitecturas de hardware modulares y sistemas operativos híbridos que combinan la planificación First Come First Serve (FCFS) y Earliest Deadline First (EDF) para optimizar la ejecución de tareas, incrementando así la robustez y fiabilidad del controlador \cite{paper} \cite{rtos}.\\

Otra área clave de investigación es la integración de sensores redundantes y la mejora en la precisión de las lecturas mediante técnicas de amortiguación mecánica. Por ejemplo, en el sistema URpilot, los sensores se distribuyen en una arquitectura de placa flexible para reducir el ruido y las vibraciones mecánicas, lo cual mejora la estimación de la actitud del UAV. Además, se están desarrollando métodos avanzados para la fusión de datos provenientes de múltiples unidades de medida inercial (IMUs) y para la detección de fallos en tiempo real, aspectos críticos para asegurar la fiabilidad del vuelo en condiciones adversas \cite{rtos}.\\

En términos de hardware, se ha logrado una optimización significativa mediante el uso de microcontroladores basados en la arquitectura ARM Cortex-M, como el STM32F767 utilizado en la placa URpilot. Este microcontrolador ofrece un alto rendimiento y capacidades de procesamiento en tiempo real, fundamentales para las aplicaciones profesionales de los UAVs. La incorporación de redundancia a nivel de sensores en esta arquitectura también mejora la precisión y fiabilidad de las mediciones críticas para el control de vuelo \cite{rtos}.\\

Recientemente, se ha propuesto un RTOS simplificado y específico para aplicaciones de UAV, con el objetivo de aumentar la velocidad de ejecución de tareas, minimizar la latencia y facilitar el uso para los programadores. Este sistema emplea un planificador híbrido en el cual las tareas críticas en tiempo real son controladas por un marco FCFS, mientras que las tareas no críticas son gestionadas por un marco EDF con prioridad dinámica. Esta configuración permite que las tareas no críticas puedan ser manejadas como tareas en tiempo real si su prioridad dinámica alcanza un valor máximo, asegurando una mayor flexibilidad y eficiencia en la gestión de las tareas del UAV \cite{rtos}.\\