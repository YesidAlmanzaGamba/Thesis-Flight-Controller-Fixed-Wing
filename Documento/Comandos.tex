%Comandos%

%********%

%Comandos de MEL: \teo, \lem, \dem, \hip, \hipp \rl, \rli, \dl, \eq, \o, \n, \i, \t, \f, \a, \o, \qd

% Para incluir el símbolo del máximo común denominador (mcd):
\newcommand{\mcd}{\text{mcd}}

% Para incluir el símbolo del mínimo común múltiplo (mcm):
\newcommand{\mcm}{\text{mcm}}

% Para incluir información del teorema. E.g., \teo{p\i q} será Teo: p => q.
\newcommand{\teo}[1]{\underline{\textbf{Teo}}:\hp\hp$\dsp\bm{{#1}} $\vspace{0.2cm}\\}

% Para crear un lema. Parámetro 1: el número o nombre del lema. Parámetro 2: la información del lema como tal. E.g., \lem{1}{p} será Lema 1: p
\newcommand{\lem}[2]{\underline{Lema {#1}}:\hp\hp$\dsp\bm{{#2}} $\vspace{0.2cm}\\}

% Representa el símbolo de subconjunto.
\newcommand{\subs}{\subseteq}

% Representa el símbolo de subconjunto propio.
\newcommand{\psubs}{\subset}

% Representa el símbolo de un conjunto vacío.
\newcommand{\es}{\varnothing}

% Para incluir el tipo de demostración. Puede dejarse vacío. Parámetro 1: el tipo de demostración. E.g., \dem{por hipótesis} será Dem: por hipótesis.
\newcommand{\dem}[1]{\underline{Dem}:\hp{#1}\vspace{0.2cm}\\}

% Para incluir los casos. Parámetro 1: los casos. E.g., \casos{c_1, c_2} será: Casos: c_1, c_2.
\newcommand{\casos}[1]{\underline{Casos}:\hp$\dsp {#1} $\vspace{0.2cm} \\}

% Para incluir un caso. Parámetro 1: el caso. E.g., \caso{p} será: Caso: p.
\newcommand{\caso}[1]{\underline{Caso}:\hp $\dsp {#1}$\vspace{0.2cm}\\}

% Para incluir las hipótesis. Parámetro 1: las hipótesis. \hip{a,b,c} será Hip: a, b, c.
\newcommand{\hip}[1]{\underline{Hip}:\hp\hp$\dsp {#1} $\vspace{0.2cm} \\}

% Para incluir los razonamientos por equivalencia. Parámetro 1: el razonamiento. E.g., \rl{De Morgan} será = < De Morgan >.
\newcommand{\rl}[1]{\small$= \hspace{0.5cm}\langle\textrm{#1}\rangle$\normalsize\\}

% Para incluir los razonamientos por equivalencia sin ningún argumento escrito. \rll será sólo =.
\newcommand{\rll}{\small$= \hspace{0.5cm}$\normalsize\\}

% Para incluir los razonamientos por algún símbolo. Parámetro 1: el símbolo. Parámetro 2: el razonamiento. E.g., \rl{De Morgan} será = < De Morgan >.
\newcommand{\rlp}[2]{\small${#1} \hspace{0.5cm}\langle\textrm{#2}\rangle$\normalsize\\}

% Para incluir los razonamientos por algún símbolo sin ningún argumento escrito. \rllp{<} será sólo <.
\newcommand{\rllp}[1]{\small${#1} \hspace{0.5cm}$\normalsize\\}

% Para incluir los razonamientos por implicación. Parámetro 1: el razonamiento. E.g., \rl{Transitividad} será => < Transitividad >.
\newcommand{\rli}[1]{\small$\Rightarrow \hspace{0.5cm}\langle\textrm{#1}\rangle$\normalsize\\}

% Para incluir los razonamientos por implicación sin ningún argumento escrito. \rlii será sólo =>.
\newcommand{\rlii}{\small$\Rightarrow \hspace{0.5cm}$\normalsize\\}

% Para incluir los procesos en modo matemáticas. Parámetro 1: el proceso. E.g., \dl{p\i a} será: p => a.
\newcommand{\dl}[1]{\vspace{0.2cm}\small$\dsp {#1} $\normalsize\\ \vspace{0.2cm}}

% El símbolo de equivalencia.
\newcommand*{\eq}{\equiv}

% El símbolo de negación.
\newcommand*{\n}{\neg\hspace{0.05cm}}

% El símbolo de implicación.
\renewcommand*{\i}{\Rightarrow}

% Para tener 'true' en letra tipo typewriter.
\renewcommand{\t}{\texttt{true}}
% Para tener 'false' en letra tipo typewriter.
\newcommand*{\f}{\texttt{false}}

% El símbolo de 'y' (and).
\renewcommand*{\a}{\footnotesize\land \small}

% El símbolo de 'o' (or).
\renewcommand*{\o}{\footnotesize\lor \small}

% Agrega un cuadrado blanco para indicar la terminación de una demostración.
\newcommand{\qd}{\hspace{0.3cm}\qed\hspace*{11cm}\textcolor{white}{ }}

% Lógica de predicados: 'para todo' con condición (3 parámetros) y sin condición (2 parámetros).
\newcommand{\fa}[3]{(\forall\hp {#1} \hp |\hp {#2} \hp : \hp {#3})}
\newcommand{\fat}[2]{(\forall\hp {#1} \hp |: {#2})}
% Lógica predicados: 'existe' con condición (3 parámetros) y sin condición (2 parámetros).
\newcommand{\ea}[3]{(\exists\hp {#1} \hp |\hp {#2} \hp : \hp {#3})}
\newcommand{\eat}[2]{(\exists\hp {#1} \hp |: {#2})}

% Lógica predicados: cuantificador de algún operador (4 parámetros) y sin condición (3 parámetros).
\newcommand{\cua}[4]{({#1}\hp {#2}\hp |\hp {#3}\hp : \hp {#4})}
\newcommand{\cuat}[3]{({#1}\hp {#2}\hp |:\hp {#3})}

% Inducción: representa el predicado de inducción.
\newcommand{\PI}[1]{\underline{PI}: $\dsp {#1}$\vspace{0.2cm}\\}

% Inducción: representa el caso base.
\newcommand{\CB}[1]{\underline{Caso base}: ${#1}$ \vspace{0.2cm}\\}

% Inducción: representa el caso inductivo.
\newcommand{\CI}[1]{\underline{Caso inductivo}: ${#1}$ \vspace{0.2cm}\\}

% Inducción: representa la hipótesis de inducción.
\newcommand{\HI}[1]{\underline{HI}: ${#1}$ \vspace{0.2cm}\\}

%********************
% Comandos no de MEL:
%********************

%Color azul:
\newcommand*{\DO}{\textcolor{DarkOrange}}

%Color azul:
\newcommand*{\DB}{\textcolor{DarkBlue}}

%Color verde:
\newcommand{\DG}{\textcolor{DarkGreen}}

%Espacio de 0.1cm:
\newcommand{\hp}{\hspace{0.1cm}}

% Para mejorar las ecuaciones.
\newcommand{\dsp}{\displaystyle}

%Para números reales \R:
\newcommand{\R}{\mathbb{R}}
%Para números naturales \N:
\newcommand{\N}{\mathbb{N}}
%Para números enteros \Z:
\newcommand{\Z}{\mathbb{Z}}

% Para incluir el símbolo de multiplicación:
\renewcommand*{\d}{\cdot}